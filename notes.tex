\documentclass{article}

\begin{document}
\title{Using B\'ezier extraction matrices}
\author{P\'eter Salvi}
\maketitle
Given a B-spline curve of degree $d$ with $L$ segments and $n$ control points in the form
\[\mathcal{C}(u)=\sum_{k=0}^{n-1}\mathbf{P}_kN_k^{d,U}(u),\]
where $U$ is the clamped knot vector with $d+1$ identical values at both ends,
and $N_k^{d,U}$ are the B-spline basis functions, the $i$-th B\'ezier component ($i=0..L-1$)
can be extracted as
\[\hat{\mathcal{C}}_i(u)=\sum_{j=0}^d\mathbf{Q}_j^iB_j^d(u),\]
where $B_j^d$ are the Bernstein polynomials and
\[\mathbf{Q}_j^i=\sum_{k=0}^d\mathbf{P}_{k+s_i-d}C^i_k,j,\]
because of the extraction relation
\[N_k^{d,U}(u)=\sum_{j=0}^dC^i_{k+d-s_i,j}B_j^d(u),\]
$\mathbf{C}^i$ being the $i$-th extraction matrix of the degree-$d$ knot vector $U$,
and $s_i$ denoting the span index of the $i$-th segment.

Conversely, a B-spline control point $\mathbf{P}_k$ can be restored from a
suitable B\'ezier segment (i.e., one that is in the same span), using the formula
\[\mathbf{P}_k=\sum_{j=0}^d\mathbf{Q}_j^i\left(\mathbf{C}^{i^{-1}}\right)_{j,k+d-s_i}.\]
\end{document}
